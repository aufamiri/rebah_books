% Ubah kalimat sesuai dengan judul dari bab ini
\chapter{TINJAUAN PUSTAKA}

\section{Kekerasan}

Kekerasan merupakan kata yang berasal dari bahasa Latin \textit{violentus} yang berarti berkuasa atau kekuasaan. Secara arti, kekerasan merupakan sebuah ekspresi baik yang dilakukan secara fisik ataupun secara verbal yang mencerminkan pada tindakan agresi dan penyerangan pada kebebasan seseorang.

Terdapat beberapa kekerasan yang dapat dibagi berdasarkan kategori - kategori tertentu, salah satu kategori - kategori tersebut adalah kekerasan yang dilakukan perorangan, atau kekerasan yang dilakukan oleh kelompok.


\section{Mobile Application}

\textit{Mobile Application} adalah program atau perangkat lunak yang dibuat untuk perangkat-perangkat bergerak seperti : \textit{Smartphone}, \textit{SmartWatch}, \textit{Tablet}, dan lainnya. Perangkat lunak atau disebut juga \textit{software} aplikasi merupakan hasil dari pemrograman \textit{mobile} yang dirancang menggunakan bahasa pemrograman tertentu. Banyak sekali keunggulan yang bisa didapatkan saat menggunakan aplikasi \textit{mobile} dibandingkan dengan aplikasi \textit{web} dan \textit{desktop}, di antaranya yaitu:
\begin{enumerate}[nolistsep]
  \item \textit{User Interface} dan \textit{User Experience} (\textit{UI/UX}) aplikasi seluler bisanya cukup menarik dan sangat mudah digunakan.

  \item Ada beberapa aplikasi yang bisa digunakan tanpa harus terkoneksi internet.

  \item Pengguna dapat mengakses aplikasi dimana saja melalui perangkat yang dimiliki.
\end{enumerate}

Aplikasi yang tidak \textit{terinstall} biasanya tersedia melalui \textit{platform} distribusi yang disebut "toko aplikasi". Mereka mulai muncul pada tahun 2008 dan biasanya dioperasikan oleh pemilik sistem operasi \textit{mobile}, seperti Apple App Store, Google Play, Windows Phone Store, dan BlackBerry App World. Namun, ada juga toko-toko aplikasi independen, seperti Cydia, GetJar dan F-Droid. Beberapa aplikasi sifatnya gratis, sementara yang lain harus dibeli. Biasanya, mereka diunduh dari \textit{platform} tersebut ke perangkat target, tetapi terkadang mereka dapat diunduh ke laptop atau komputer \textit{desktop}. Untuk aplikasi berbayar, umumnya sebesar 20-30\% untuk penyedia distribusi (seperti iTunes), dan sisanya masuk ke produsen aplikasi. Oleh karena itu, aplikasi yang sama dapat dikenakan biaya yang berbeda tergantung pada \textit{desktop}nya.
\vspace{0.5ex}

Penggunaan aplikasi seluler menjadi semakin lazim bagi para pengguna ponsel. Sebuah studi comScore pada Mei 2012 melaporkan bahwa selama kuartal sebelumnya, pelanggan perangkat \textit{mobile} lebih banyak menggunakan aplikasi dibanding menjelajahi \textit{web} di perangkat mereka: masing-masing 51,1\% vs 49,8\%. Para peneliti menemukan bahwa penggunaan aplikasi \textit{mobile} sangat berkorelasi dengan konteks pengguna dan bergantung pada lokasi dan waktu pengguna.

\section{Figma}

Figma adalah perangkat lunak yang digunakan oelh para desainer aplikasi \textit{mobile} maupun desainer \textit{web} untuk mendesain suatu UI / UX dan bersifat \textit{web-based} dimana maksudnya adalah Figma dapat diakses dimana saja selama terdapat \textit{browser} dan koneksi internet. Fitur - fitur utama dari Figma adalah kemudahan dalam melakukan desain UI maupun UX, dan memiliki kemampuan untuk melakukan kolaborasi secara \textit{real-time} sehingga dapat mempermudah tim yang berkolaborasi secara \textit{remote}. Selain itu, sudah disediakan banyak sekali \textit{template} sehingga sangat memudahkan dalam membuat tampilan yang bagus dan menarik dari awal.

\section{Firebase}

Firebase adalah \textit{platform} yang dibuat oleh Google yang bertujuan untuk memudahkan pengembang perangkat lunak dalam membuat aplikasi gawai dan \textit{web}. Terdapat banyak sekali fitur - fitur yang ditwarkan seperti \textit{Cloud Firestore} yang merupakan basis data NoSQL yang ditawarkan oleh Google, \textit{Authentication} yang berguna untuk melakukan autentikasi secara mudah dan sudah mendukung beberapa \textit{platform} seperti Email dan Password, Google SignIn, Facebook SignIn, dan masih banyak lagi, kemudian terdapat \textit{Cloud Messaging} yang digunakan untuk mengirimkan notifikasi kepada \textit{device} seperti Android, maupun IOS. Dan masih banyak lagi fitur - fitur yang ditawarkan pada Firebase ini.

\section{App Engine}

App Engine adalah \textit{platform} dari Google yang berada dibalik bendera \textit{Google Cloud Platform (GCP)}. \textit{Google Cloud Platform} sendiri adalah gabungan dari beberapa \textit{cloud service} yang dimiliki oleh Google dan \textit{server} nya sudah tersebar di berbagai negara di dunia.

App Engine sendiri adalah sebuah \textit{serverless platform} yang bersifat \textit{fully managed.} Yang dimaksud sebagai \textit{fully managed} adalah ketika \textit{server} atau \textit{service} sepenuhnya diurus oleh pihak Google sehingga kita sebagai pengembang perangkat lunak hanya perlu fokus pada perangkat lunak yang kita buat dan tidak terganggu oleh hal - hal lain seperti konfigurasi \textit{server}, memilih \textit{database}, dan hal - hal yang menyangkut infrastruktur lainnya. Selain itu, fiturnya sendiri sudah cukup lengkap, seperti \textit{auto-scaling} yaitu fitur yang berguna apabila pengguna perangkat lunak kita meningkat maka akan secara otomatis App Engine akan menaikkan model \textit{server} yang digunakan, begitu juga sebaliknya, saat pengguna perangkat lunak kita turun maka akan secara otomatis kita akan 'dipilihkan' konfigurasi yang paling kecil dan bahkan bisa saja \textit{service} kita dimatikan sementara sampai ada yang mengakses perangkat lunak kita.