% Ubah kalimat sesuai dengan judul dari bab ini
\chapter{PENDAHULUAN}

% Ubah konten-konten berikut sesuai dengan yang ingin diisi pada bab ini

\section{Latar Belakang}

Kekerasan merupakan kata yang berasal dari bahasa Latin \textit{violentus} yang berarti berkuasa atau kekuasaan. Secara arti, kekerasan merupakan sebuah ekspresi baik yang dilakukan secara fisik ataupun secara verbal yang mencerminkan pada tindakan agresi dan penyerangan pada kebebasan seseorang.

Menurut data yang didapat dari Komisi Nasional (Komnas) Perempuan, terdapat lebih dari 8000 kasus yang sudah ditangani hanya dalam kurun waktu tahun 2020 saja. Dan yang lebih parah lagi, apabila dibandingkan dengan tahun sebelumnya, terjadi kenaikan sebesar 60\%  kasus. Jumlah tersebut hanya menghitung jumlah kasus yang tercatat dan diketahui, namun, menurut perkiraan Komnas Perempuan, masih banyak sekali kasus yang tidak tercatat dan diketahui karena meningkatnya jumlah kuesioner yang dikembalikan, padahal kuesioner sendiri merupakan alat bantu yang cukup vital untuk mengetahui kondisi lingkungan sekitar pada seorang perempuan atau anak.

Hal yang sama juga terjadi di dalam sekolah. Data dari unicef menunjukkan bahwa pada tahun 2018, hampir 41\% anak pada umur 15 tahun pernah mengalami perundungan atau yang biasa disebut \textit{bullying}. Mirisnya, perundungan tidak hanya terbatas dilakukan oleh sesama siswa, namun tidak jarang juga dilakukan oleh guru.

Ditambah lagi dengan kondisi pandemi seperti sekarang ini, banyak sekali orang yang beraktifitas dari rumah, dan karena perubahan yang begitu tiba - tiba, banyak membuat orang menjadi stress dan tertekan. Imbasnya, semakin meningkatkan kemungkinan terjadinya kekerasan di dalam ruangan.

Salah satu hal yang paling menyulitkan dalam menangani kekerasan pada perempuan dan anak adalah karena seringnya, kekerasan dilakukan di area tertutup sehingga tidak diketahui oleh orang lain. Tidak jarang juga korban kekerasan banyak yang tidak mau melapor kepada petugas yang terkait, hal ini dibuktikan dengan banyaknya kuesioner Komnas Perempuan yang dikembalikan.

Sebagai mahasiswa departemen Teknik Komputer, saya berupaya merealisasikan sebuah sistem yang bernama Jaga Bersama. Jaga Bersama adalah sebuah sistem \textit{monitoring} cerdas yang dapat mendeteksi apabila terjadi suatu kekerasan, terkhusus kekerasan di dalam ruangan. Diharapkan dengan terealisasikannya sistem kami, maka dapat membantu untuk mengurangi angka kekerasan pada perempuan dan anak di Indonesia nantinya.


\section{Tujuan}

Tujuan dari magang ini dapat dilihat dari dua
sudut pandang sebagai berikut:

\vspace{0.5ex}

\begin{enumerate}[nolistsep]

      \item \textit{Secara Umum}
            \vspace{0.5ex}

            \begin{enumerate}[nolistsep]
                  \item Terciptanya suatu hubungan yang sinergis dan terarah antara dunia
                        perguruan tinggi dan dunia kerja sebagai pengguna outputnya.

                  \item Membuka wawasan mahasiswa agar dapat mengetahui dan memahami
                        aplikasi ilmunya di dunia industri.

                  \item Mahasiswa dapat mampu mengadakan pendekatan masalah secara utuh.

                  \item Menciptakan pola berpikir konstruktif yang
                        lebih berwawasan bagi mahasiswa.

            \end{enumerate}

            \vspace{0.5ex}

      \item \textit{Secara Khusus}
            \vspace{0.5ex}

            \begin{enumerate}[nolistsep]
                  \item Memperdalam pengetahuan mahasiswa dengan mengenal dan juga mempelajari
                        secara langsung mengenai pengolahan \textit{mobile application development} dan IoT secara umum.

                  \item Mengembangkan pengetahuan, sikap, keterampilan, kemampuan profesi melalui
                        penerapan ilmu, latihan kerja, dan pengamatan teknik yang akan diterapkan.
            \end{enumerate}

\end{enumerate}

\section{Bentuk Kegiatan}

Dalam proses magang, kami melaksanakan di rumah masing - masing (\textit{Work From Home}) dikarenakan program yang dilakukan di seluruh Indonesia secara serentak sehingga memang harus dilakukan secara jarak jauh melalui media Google Meet.

Pembuatan sistem kami kerjakan sesuai dengan \textit{timeline} yang sudah kami sepakati dan kami mengerjakan sesuai peran masing - masing. Adapun bentuk teknis kegiatannya adalah sebagai berikut :

\begin{enumerate}[nolistsep]
      \item Penyusunan laporan sementara yang dilakukan pada minggu terakhir kegiatan
            magang dilaksanakan. Selanjutnya laporan sementara ini dimintakan persetujuan
            kepada pembimbing magang yang bersangkutan. Dalam memberikan persetujuan
            terhadap laporan sementara ini maka pembimbing lapangan memberikan penilaian terhadap
            pelaksanaan magang yang dilaksanakan oleh mahasiswa. Laporan sementara ini
            selanjutnya dijadikan dasar untuk pembuatan laporan resminya. Penilaian dari pembimbing
            lapangan akan diserahkan pada dosen pembimbing magang di Departemen Teknik
            Komputer - ITS Surabaya.

      \item Penilaian kedua diberikan berdasarkan hasil penyusunan laporan resmi yang dibuat
            oleh mahasiswa selesai dilaksanakannya magang.

      \item Penilaian ketiga diberikan berdasarkan hasil presentasi laporan resmi magang.
\end{enumerate}

\section{Waktu dan Tempat Pelaksanaan}

Berikut tempat dan waktu pelaksanaan magang yang dilakukan pada :

\begin{tabular}{c l}
      Tempat & : Rumah Masing - Masing (WFH)  \\
      Waktu  & : 03 Mei 2021 s/d 09 Juni 2021
\end{tabular}

\section{Metodologi Magang}

Metodologi magang yang digunakan dalam pelaksanaan magang ini adalah sebagai berikut :

\begin{enumerate}[nolistsep]

      \item \textbf{Tahap Persiapan}

            Membuat \textit{timeline} beserta proposal untuk kemudian akan dicek dan diberi kritik atau saran oleh tim Bankgit.

      \item \textbf{Pengembangan Sistem}

            Sistem dikembangkan sesuai ide awal yang sudah tertulis di dalam proposal yang sudah disetujui. Selain itu juga dilakukan \textit{mentoring} dengan orang - orang dari industri.

      \item \textbf{Pembuatan Laporan}

            Pembuatan laporan yang dibutuhkan sebagai prasayarat kelulusan dalam mata kuliah magang.

\end{enumerate}

\section{Sistematika Penulisan}

Laporan Kerja Praktek ini terdiri atas lima bab dengan rincian sebagai berikut :
\vspace{0.5ex}

\begin{enumerate}[nolistsep]

      \item \textbf{Bab I Pendahuluan}
            \vspace{0.5ex}

            Pada BAB I dibahas mengenai latar belakang, tujuan, waktu dan tempat
            pelaksanaan Kerja Praktek, metode penulisan, serta sistematika penulisan.
            \vspace{0.5ex}

      \item \textbf{Bab II Profil Google Bangkit}
            \vspace{0.5ex}

            Pada BAB II dibahas mengenai profil singkat dari program Google Bangkit.
            \vspace{0.5ex}

      \item \textbf{Bab III Tinjauan Pustaka}
            \vspace{0.5ex}

            Pada BAB III dibahas mengenai teori-teori penunjang dalam pembuatan produk aplikasi,
            seperti \textit{mobile application}, adobe XD, \textit{firebase} dan \textit{App Engine}.
            \vspace{0.5ex}

      \item \textbf{Bab IV Desain dan Implementasi}
            \vspace{0.5ex}

            Pada BAB IV dibahas mengenai pembuatan produk.

            \vspace{0.5ex}

      \item \textbf{Bab V PENUTUP}
            \vspace{0.5ex}

            Pada BAB V dibahas mengenai kesimpulan dan saran.
            \vspace{0.5ex}

\end{enumerate}