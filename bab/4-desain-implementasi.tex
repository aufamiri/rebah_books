% Ubah kalimat sesuai dengan judul dari bab ini
\chapter{DESAIN DAN IMPLEMENTASI}

% Ubah konten-konten berikut sesuai dengan yang ingin diisi pada bab ini

\section{Deskripsi Sistem}

Sistem akan dibuat dengan \lipsum[21][1-12]

\section{Implementasi Alat}

Alat diimplementasikan dengan \lipsum[22]

% Contoh pembuatan code snippet
\begin{lstlisting}[
  language=C++,
  label={lst:Hello World},
  caption={Program hello world}
]
#include <iostream>

int main() {
    std::cout << "Hello World!";
    return 0;
}
\end{lstlisting}

% Contoh penggunaan referensi dari code snippet yang diinputkan
Seperti contoh pada baris program Listing \ref{lst:Hello World} dan Listing \ref{lst:PrimeNumber}, \lipsum[23]

% Contoh input code snippet
\lstinputlisting[
  % Bahasa yang digunakan oleh code snippet
  language=Python,
  % Label referensi dari code snippet yang diinputkan
  label={lst:PrimeNumber},
  % Keterangan dari code snippet yang diinputkan
  caption={Program perhitungan bilangan prima}
% Nama dari file code snippet yang diinputkan
]{program/prime-number.py}
