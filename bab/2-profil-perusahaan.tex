% Ubah kalimat sesuai dengan judul dari bab ini
\chapter{PROFIL PROGRAM GOOGLE BANGKIT}

% Ubah konten-konten berikut sesuai dengan yang ingin diisi pada bab ini

\section{Sejarah Google Bangkit}

Google Bangkit adalah sebuah program yang digagas oleh google. Diawali pada tahun 2020 dengan berisi 300 peserta, Google Bangkit berhasil meraih perhatian banyak orang termasuk, menteri pendidikan saat ini, yaitu Nadiem Makarim. Banyak sekali alumni - alumninya yang mengatakan bahwa dengan mengikuti program Google Bangkit ini, mereka mendapatkan prospek kerja yang lebih baik sehingga meningkatkan kemungkinan untuk mendapatkan pekerjaan.

Pada tahun 2021 ini, Google Bangkit menerima peserta dengan jumlah yang jauh lebih banyak, yaitu sebesar 3000 peserta. Berbeda dengan Google Bangkit pada tahun 2020, tahun ini terdapat 3 fokusan yang dapat dipilih oleh peserta, terdapat \textit{Google Cloud Engineer}, \textit{Android Engineer} dan \textit{Machine Learning Engineer}. Tidak berhenti sampai disana, saat ini Google Bangkit berkerja sama dengan 15 partner universitas dan bahkan berkolaboasi dengan perusahan - perusahaan \textit{unicorn} seperti Gojek, Tokopedia, dan Traveloka.

Tentu saja disini sebagai suatu perusahaan yang sangat ternama, Google tidak hanya mengambil peserta dari kota - kota besar. Menurut William Florence, Pemimpin \textit{Asia Pacific Education Program}, bahwa terdapat 3000 peserta yang berasal dari 250 universitas dari segala penjuru Indonesia.  Selain itu, terdapat 30\% peserta perempuan, yang apabila dibandingkan dengan tahun lalu, terjadi peningkatan sebesar 4\%.

\section{Tujuan Google Bangkit}

Tujuan utama dari program Google Bangkit yang utama adalah meningkatkan kualitas manusia Indonesia sehingga menjadi manusia yang kreatif, terampil dan ahli dalam bidangnya masing - masing.

Selain itu, melalui program ini, diharapkan mahasiswa dapat membangun kepercayaan diri dalam berekreasi. Hal ini termasuk dengan kemampuan bertindak dan kerangka pikir kewirausahaan agar dapat mengatasi tantangan global dan membangun masa depan yang lebih baik.

Dan apabila kita melihat lebih dalam apa saja yang disuguhkan oleh Google Bangkit, dapat dilihat bahwa mahasiswa juga diharapkan untuk mengasah ketrampilan penting yang bermanfaat dalam dunia kerja, seperti \textit{design thinking}, kepemimpinan, serta kemampuan komunikasi dan presentasi.


