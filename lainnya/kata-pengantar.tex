\begin{center}
  \Large\textbf{KATA PENGANTAR}
\end{center}
\vspace{2ex}

\addcontentsline{toc}{chapter}{KATA PENGANTAR}

Puji syukur atas kehadirat Tuhan yang Maha Esa, karena atas rahmat dan karunia-Nya penulis telah dapat menyelesaikan magang di Google Bangkit yang dilaksanakan tanggal 1 Mei 2021 sampai dengan 09 Juni 2021. Dalam penyelesaian Laporan Magang ini, kami mengucapkan terimakasih kepada berbagai pihak yang telah membantu dalam penyelesaian laporan ini :

\begin{itemize}
  \item Bapak Dr. Supeno Mardi Susiki Nugroho, ST.,MT. selaku Kepala Departemen Teknik Komputer FTEIC-ITS.
  \item Dr. Diah Puspito Wulandari, S.T.,M.Sc. selaku Koordinator Magang Departemen Teknik Komputer ITS.
  \item Bapak Fuad Rachmadi selaku dosen pembimbing dalam magang ini.
  \item Dian Ayuningtyas sebagai mentor sekaligus pembimbing selama melakukan proses magang.
\end{itemize}

Magang merupakan kredit mata kuliah opsional yang digunakan sebagai konversi dalam Google Bangkit. Magang juga berfungsi sebagai pendalaman mahasiswa dalam pengaplikasian ilmu yang didapat pada saat perkuliahan ke dalam dunia kerja. Penulis menyampaikan permohonan maaf jika selama pelaksanaan magang terdapat hal yang kurang berkenan dan jikalau ada salah dalam penulisan laporan ini.

\begin{flushright}
  \begin{tabular}[b]{c}
    % Ubah kalimat berikut sesuai dengan tempat, bulan, dan tahun penulisan
    Surabaya, Juni 2021
    \\
    \\
    \\
    \\
    Penulis
  \end{tabular}
\end{flushright}
